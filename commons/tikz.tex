% !TEX root = main.tex
%% TikZ und Co.
\usepackage{tikz} % Grafiken
\usetikzlibrary{babel}
\usetikzlibrary{arrows}
\usetikzlibrary{arrows.meta}
\usetikzlibrary{automata}
\usetikzlibrary{positioning}
\usetikzlibrary{matrix}
\usetikzlibrary{fit}
\usetikzlibrary{backgrounds}
\usetikzlibrary{shadows}
\usetikzlibrary{shapes.multipart}
\usetikzlibrary{shapes.misc}
\usetikzlibrary{shapes.geometric}
\usetikzlibrary{shapes.arrows}
\usetikzlibrary{shapes.symbols}
\usetikzlibrary{shapes.callouts}
\usetikzlibrary{er}
\usetikzlibrary{shadings}
\usetikzlibrary{topaths}
\usetikzlibrary{chains}
\usetikzlibrary{decorations.pathreplacing}
\usetikzlibrary{decorations.pathmorphing}
\usetikzlibrary{patterns}
\usetikzlibrary{patterns.meta}
\usetikzlibrary{calc}
\usetikzlibrary{mindmap}
\usetikzlibrary{through}

\usepackage[linguistics]{forest}

\usepackage{pgf,pgfplots,etoolbox}
\pgfplotsset{compat=newest}

%% background incompatibility with dependency package:
% \usepackage{tikz-dependency}

%% some TikZ styles for usage or as template:
\tikzset{
    obj/.style={
        draw=SeminarGrau, double=white, anchor=text, rectangle split, rectangle split parts=4, rectangle split part align={left}, rectangle split part fill={SeminarBlau, SeminarGruen, SeminarGrau, SeminarRot}, text=white
    },
    ent/.style={
        draw=SeminarGrau, double=white, anchor=text, rectangle split, rectangle split parts=3, rectangle split part align={center}, rectangle split part fill={SeminarBlau, SeminarGruen, SeminarGrau}, text=white
    },
    overtag/.style={
        font=\normalsize,fill=white
    },
    tag/.style={
        font=\normalsize
    },
    kuller/.style={
        -o
    },
    dreibein/.style={
        -angle 90 reversed, shorten >=1pt
    },
    kullerdreibein/.style={
        o-angle 90 reversed, shorten >=1pt
    },
    dreibeinB/.style={
        angle 90 reversed-angle 90 reversed, shorten >=1pt, shorten <=1pt
    },
    kreis/.style={
        circle, draw, fill=white, inner sep=0pt, minimum width=4pt
    },
    association/.style={
        ellipse split, fill=LNlightblue, draw=SeminarGrau, double=white, drop shadow, text=black, inner sep=2pt
    },
    isa/.style={
        semicircle, fill=LNlightblue
    },
    phenomenon/.style={
        draw=SeminarGrau, double=white, drop shadow, fill=LNshadowblue, text=white, align=center, minimum height=1.6cm, text width=3.3cm
    },
    modell/.style={
        draw=SeminarGrau, double=white, drop shadow, fill=SeminarGruen, text=white, align=center, minimum height=1.6cm, text width=3.3cm
    },
    formalism/.style={
        draw=SeminarGrau, double=white, drop shadow, fill=black, text=white, align=center, minimum height=1.6cm, text width=3.3cm
    },
    instanziiert/.style={
        stealth-, font=\footnotesize, thick
    },
    eLexTable/.style={
        draw=SeminarGrau, double=white, anchor=text, rectangle split, rectangle split parts=3, rectangle split part align={center}, rectangle split part fill={SeminarGrau!40, LNlightblue, white}, text=black
    },
    some/.style={
        circle, draw=SeminarGrau, double=white, drop shadow, fill=LNshadowblue, text=white, align=center
    },
    korpus/.style={
        draw=SeminarGrau, double=white, fill=SeminarGruen, text=white, tape, tape bend top=none, text width=1.4cm, minimum height=2.1cm, align=center, double copy shadow={draw=SeminarGrau, thick, opacity=0.6, shadow xshift=1ex, shadow yshift=1ex}
    },
    aufbereitung/.style={
        rectangle, draw=SeminarGrau, double=white, drop shadow, fill=#1, text=white, text width=2.3cm, minimum height=1.2cm, align=center
    },
    schnittstelle/.style={
        circle, draw=SeminarGrau, fill=SeminarGrau, double=white, drop shadow, minimum size=1cm
    },
    instanz/.style={
        fill=SeminarRot,text=white, rectangle, thick, draw=SeminarGrau, font=\footnotesize
    },
    modell/.style={
        chamfered rectangle, draw=SeminarGrau, double=white, drop shadow, text=white, chamfered rectangle sep=0.2cm, fill=SeminarBlau, chamfered rectangle corners=north west, text width=1.8cm, minimum height=2.3cm, align=flush left
    },
    dokument/.style={
        cylinder, shape border rotate=90, aspect=0.25, fill=SeminarBlau, draw=SeminarGrau, double=white, text=white, text width=1.4cm, align=center, drop shadow, minimum height=2.3cm
    },
    eLex/.style={
        loose background, top color=SeminarGruen!40, bottom color=LNlightblue, middle color=SeminarGrau!20,  draw=SeminarGrau
    },
    defnode/.style={
        rectangle, rounded corners, drop shadow, fill=LNultralightblue, text width=11cm, align=flush left
    },
    zitterkreis/.style={
        draw=black, decorate, decoration={random steps, segment length=3pt, amplitude=0.5pt}, circle, line width=1.5pt, minimum size=1.1cm
    },
    rectext/.style={
        rectangle, font=\TTBOLD, fill=white, align=center
    },
    onlytext/.style={
        font=\TTBOLD, fill=white, align=center
    },
    zitterquadrat/.style={
        rectangle, draw=black, line width=1.5pt, decorate, decoration={random steps, segment length=3pt, amplitude=0.4pt}, minimum size=1.3cm
    },
    zitterlinie/.style={
        draw, thick, decoration={random steps, segment length=15pt, amplitude=0.75pt}
    },
    rectangular vjoint/.style={
        to path={-- ++(0,-0.6) -| (\tikztotarget)}
    },
    rectangular hjoint/.style={
        to path={-- ++(-0.6,0) -| (\tikztotarget)}
    },
    Hinweis/.style={
        circle, fill=SeminarRot, align=center, text width=2cm, text=white
    }
}

\RequirePackage{pgfplots}
\pgfplotsset{%
    barplotTable/.style={
        nodes near coords*={$\pgfmathprintnumber\y$},
        visualization depends on={\thisrow{y} \as \y},
        every node near coord/.append style={color=black, rotate=50, yshift=1pt, xshift=10pt, /pgf/number format/.cd, precision=1}
    },
    discard if/.style 2 args={
        x filter/.code={
            \edef\tempa{\thisrow{#1}}
            \edef\tempb{#2}
            \ifx\tempa\tempb
            \def\pgfmathresult{inf}
            \fi
        }
    },
    discard if not/.style 2 args={
        x filter/.code={
            \edef\tempa{\thisrow{#1}}
            \edef\tempb{#2}
            \ifx\tempa\tempb
            \else
            \def\pgfmathresult{inf}
            \fi
        }
    }
}
